Open\+G\+L 3.\+3 visualization of stars with \hyperlink{classOculus}{Oculus} Rift mode.

Research and Development Project for the Observatoire Astronomique de Strasbourg. Tested on Linux but should work on other plateforms too with little to no changes.

This project is in no means a model to follow if you\textquotesingle{}re developping an Open\+G\+L application, but is rather a proof-\/of-\/concept and demonstration of what you can do with the \hyperlink{classOculus}{Oculus} Rift.

\subsection*{Install}


\begin{DoxyItemize}
\item Clone the repo
\item {\ttfamily git submodule init} and {\ttfamily git submodule update}
\item Update your graphic card drivers
\item Install the \hyperlink{classOculus}{Oculus} S\+D\+K 0.\+3.\+2 for your plateform (the files included are for the Linux \hyperlink{classOculus}{Oculus} S\+D\+K).
\item Install the S\+D\+L2 library
\item Install the S\+D\+L2 Image library
\item Install the Boost Program options library (tested with 1.\+55)
\item Make sure you have a c++11 compliant compiler (clang works fine).
\item Compile (either qmake or make -\/ the Makefile here was generated by qmake for Linux)
\item Plug-\/in your \hyperlink{classOculus}{Oculus} Rift if you have one (tested with D\+K1, should work with D\+K2)
\item Launch (./\+Simulation)
\end{DoxyItemize}

\subsection*{Command line options}

``` -\/h \mbox{[} --help \mbox{]} Produce help message -\/o \mbox{[} --oculus \mbox{]} \hyperlink{classOculus}{Oculus} mode -\/f \mbox{[} --fullscreen \mbox{]} Fullscreen mode -\/t \mbox{[} --texture \mbox{]} arg (=Textures/photorealistic/photorealistic\+\_\+marble/granit01.\+jpg) Set the texture used on the cubes -\/n \mbox{[} --number \mbox{]} arg (=1024) Set the number of objects seen -\/s \mbox{[} --size \mbox{]} arg (=128) Set the size of the data cube. Must be a power of 2 --octant\+Size arg (=8) Set the size of an octant. Must be a power of 2 -\/d \mbox{[} --octant\+Drawn\+Count \mbox{]} arg (=2) Set the number of octant drawn count. 1 to only draw the octant the camera is currently in, 2 to draw the immediate neighbors, ...

```

Examples\+:

``` ./\+Simulation -\/t Textures/photorealistic/photorealistic\+\_\+marble/granit08.\+jpg ./\+Simulation -\/t Textures/photorealistic/photorealistic\+\_\+marble/granit06.\+jpg -\/n 1000 ./\+Simulation ./\+Simulation -\/h ./\+Simulation -\/d 1 ./\+Simulation -\/d 2 ./\+Simulation -\/s 64 ./\+Simulation --octant\+Size 4

```

\subsection*{How it works}

The scene is an Octree which stores the objects. Only the octant we are actually in is displayed. By tweaking the value of a parameter you can also display the neighbour octants.

\subsection*{Performance}

On my computer (16 Gb R\+A\+M, 1 Gb V\+R\+A\+M, R\+A\+D\+E\+O\+N H\+D 8570) it runs at around 30 F\+P\+S constant. The initial generation takes around 4s for 1024 objects and is linear in the number of objects.

\subsection*{Documentation}

Type {\ttfamily doxygen} in console and it should generate the documentation following the {\ttfamily Doxyfile} file.

\subsection*{Logs}

Logs can be filtered. Have a look at {\ttfamily log.\+log} and {\ttfamily log.\+err}. 